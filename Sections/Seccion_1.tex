\clearpage
\section{Consideraciones}

\subsection{Justificación}
\label{sec:Intro}
El siguiente trabajo fue llevado a cabo en la Pontificia Universidad Javeriana de Bogotá D.C, en colaboración con los profesores Edwin Munévar y Alfonso Leyva. El estudio del ADN es fundamental hoy día, comprender y evaluar el daño de este, junto con sus subsecuentes efectos biológicos tiene un sin fin de aplicaciones en las que se encuentran el tratamiento del cáncer, la protección radiológica, modificación de genes, etc. Este trabajo se basa en la interacción del ADN con la radiación y de su estudio con herramientas computacionales con el fin de entender más de como son las interacciones y los diversos fenómenos que acompañan.
\subsection{Objetivos}
\subsubsection{Objetivo General}
\begin{itemize}
  \item Estudiar el efecto de la radiación ionizante sobre los enlaces químicos que definen la estructura tridimensional de una macromolécula de ADN.
\end{itemize}
\subsubsection{Objetivos Específicos}
\begin{itemize}
  \item A partir de un código base “Geant4-DNA” estructurar un programa que permita realizar apropiadamente la simulación del ADN y su interacción con radiación ionizante.
  \item Basados en los resultados de la simulación se determina los efectos de la radiación sobre la estructura misma de la bio-macromolécula
  \item Mediante el análisis de los resultados obtenidos, tener la debida dinámica propia de la bio-macromolécula luego de haber sido irradiada.
\end{itemize}
\subsubsection{Resultados alcanzados}
En base al proyecto de grado junto con su debido cronograma de actividades se obtuvieron los siguientes resultados:
\begin{itemize}
  \item Se construyo un referente bibliográfico	
a partir de diferentes fuentes de información tales como
artículos, libros y manuales, con el fin de establecer los conceptos necesarios y tener la información adecuada.
\item Se realizo el debido entrenamiento del software, entre lo que se encuentra el uso de ROOT para generar histogramas, VMD(\textbf{V}isual \textbf{M}olecular \textbf{D}inamics) un software desarrollado por la universidad Illinois para la visualización de moléculas, GEANT4 toolkit para el estudio de partículas cargadas a través de la materia y por ultimo Gromacs un software encargado de la simulación de la dinámica molecular de sistemas en los que se encuentran proteínas, lípidos y ácidos nucleicos.
\item Se estudio y se modificó un ejemplo muy específico y complejo de Geant4 conocido como pdb4dna, de este se obtuvieron diversos resultados para el análisis de una macromolécula de ADN.
\end{itemize}
