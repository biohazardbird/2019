\clearpage
\section{Conclusiones}
\label{sec:res}
Se ha mostrado que pdb4dna es una aplicación bastante interesante de Geant4 la cual permite realizar cálculos de deposición de energía y conteos tanto de rompimientos simples como dobles de ciertas moléculas de ADN mediante un archivo pdb con ciertas condiciones, bien la aplicación puede ser extendida al uso de moléculas de ADN más complejas y también otros tipos de moléculas, a partir de la energía depositada se presenta la posibilidad de modificar el código de manera que sabiendo donde se deposita la energía y el filtro poder realizar una localización de los rompimientos simples y dobles en la molécula de ADN lo cual permitiría el uso de Ggromacs para estudiar la dinámica propia de está.\\
En otra instancia también se modificó Geant4 para que usara centros de masa en lugar de esferas ligadas basadas en baricentros, al realizar histogramas tanto de las esferas ligadas y los baricentros no se ve ninguna diferencia en la deposición de energía ni en los rompimientos simples como dobles bajo las mismas condiciones de eventos, partículas, y energía. Esto se podría deber a que el código no está tomando correctamente los centros de masa nuevos o a que el valor de la magnitud de las masas no afecta de forma relevante las posiciones del centro de masa respecto a las de los baricentros, sin embargo de ser la segunda opción esto podría cambiar de forma relevante con moléculas mucho más complejas o grandes debido a que como se ve en los histogramas de las diferentes posiciones de centros de masa contra baricentros se observan cambios a simple vista lo que resultaría en un cambio de la deposición de energía y en consecuencia de los rompimientos y posterior dinámica estructural, lo anterior se hizo con el fin de proponer un modelo alterno diferente a las esferas ligadas en baricentros con el fin de que sea más acercado al modelo general de átomos y masa, por otro lado está la modificación del código para que en lugar de tomar la deposición de energía en fosfato-azúcar, sea en azúcar-base, esto con el fin de que sea posible realizar un estudio en diferentes tipos de enlaces, esto tiene varios resultados, el primero es que el programa se podría ampliar mucho más lo cual significaría un avance en muchos más cometidos, por otro lado a partir de poder evaluar los rompimientos en las bases se podría estudiar los daños causados a estas y sus subsecuentes efectos.
